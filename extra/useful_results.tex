\documentclass[12pt]{article}

%\documentclass{amsart}
%\documentclass{scrartcl}
%\usepackage{changepage}
%\usepackage{scrextend}

\usepackage{amssymb,amsmath,amsthm}
% amssymb has empty set symbo
\usepackage{scrextend} % for \begin{addmargin}[0.55cm]{0cm} text \end{margin}

\usepackage{mathrsfs} % for \mathscr{P}
\usepackage{float}
\usepackage{enumitem}
\usepackage{hanging}
\usepackage{verbatim}
\usepackage[colorlinks,linkcolor=blue]{hyperref}
\usepackage[nameinlink]{cleveref}

\newcommand{\n}{ \noindent }
\newcommand{\F}{\mathcal{F}}
\newcommand{\G}{\mathcal{G}}


\newcommand{\pwset}{\mathcal{P}}
%\newcommand{\pwset}{\mathscr{P}}


\newtheorem*{theorem}{Theorem}  % This enables \begin{theorem}

%\DeclareFontFamily{U}{MnSymbolC}{}
%\DeclareSymbolFont{MnSyC}{U}{MnSymbolC}{m}{n}
%\DeclareFontShape{U}{MnSymbolC}{m}{n}{
%  <-6>    MnSymbolC5
%  <6-7>   MnSymbolC6
%  <7-8>   MnSymbolC7
%  <8-9>   MnSymbolC8
%  <9-10>  MnSymbolC9
%  <10-12> MnSymbolC10
%  <12->   MnSymbolC12%
%}{}
%\DeclareMathSymbol{\powerset}{\mathord}{MnSyC}{180}

%\usepackage[left=2cm,right=2cm,top=2cm,bottom=2cm]{geometry}
\usepackage[left=1.2in, right=1.2in, top=1in, bottom=1in]{geometry}

\newtheorem*{proposition}{Proposition}
\newtheorem{lemma}{Lemma}

\newcommand{\lemmaautorefname}{Lemma}

\title{Useful Results}

\begin{document}

\maketitle

Below are some results I proved in my Introduction to Real Analysis Course. I found some of them useful and even though some may seem obvious I wanted to do the formal proof for myself. Also, I wanted a place to put these results so I could come back and find them pretty easily when needed.


\begin{lemma} \label{lemma:a}
If $a \in \mathbb{R}$, $\varepsilon \in \mathbb{R}$, and $\varepsilon > 0$, then $a < a + \varepsilon$.
\end{lemma}

\begin{proof}
Let $a, \varepsilon \in \mathbb{R}$ and suppose $\varepsilon > 0$. For the sake of contradiction assume $a \geq a + \varepsilon$. Then $a - (a + \varepsilon) \geq 0$, which implies that $-\varepsilon \geq 0$ and so $\varepsilon \leq 0$. However, $\varepsilon > 0$ by assumption and so we have that $\varepsilon > 0$ and $\varepsilon \leq 0$. This is a contradiction because for every real number $x$ we have $x < 0$, $x = 0$, or $x > 0$. Therefore our assumption that $a \geq a + \varepsilon$ must be false and so we can conclude $a < a + \varepsilon$. 
\end{proof}

\begin{lemma}
If $a, b \in \mathbb{R}$ and $a < b$, then $\displaystyle a < a + \frac{|b - a|}{2} < b$.
\end{lemma}

\begin{proof}
Let $a, b \in \mathbb{R}$ and suppose $a < b$. Since $\displaystyle \frac{|b - a|}{2} > 0$ then by \autoref{lemma:a} $\displaystyle a < a + \frac{|b - a|}{2}$. 
\end{proof}




\end{document}