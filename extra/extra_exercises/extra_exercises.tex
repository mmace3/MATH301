\documentclass[12pt]{article}

%\documentclass{amsart}
%\documentclass{scrartcl}
%\usepackage{changepage}
%\usepackage{scrextend}

\usepackage{amssymb,amsmath,amsthm}
% amssymb has empty set symbo
\usepackage{scrextend} % for \begin{addmargin}[0.55cm]{0cm} text \end{margin}

\usepackage{mathrsfs} % for \mathscr{P}
\usepackage{float}
\usepackage{enumitem}
\usepackage{hanging}
\usepackage{verbatim}
\usepackage[colorlinks,linkcolor=blue]{hyperref}
\usepackage[nameinlink]{cleveref}

\newcommand{\n}{ \noindent }
\newcommand{\F}{\mathcal{F}}
\newcommand{\G}{\mathcal{G}}


\newcommand{\pwset}{\mathcal{P}}
%\newcommand{\pwset}{\mathscr{P}}


\newtheorem*{theorem}{Theorem}  % This enables \begin{theorem}

%\DeclareFontFamily{U}{MnSymbolC}{}
%\DeclareSymbolFont{MnSyC}{U}{MnSymbolC}{m}{n}
%\DeclareFontShape{U}{MnSymbolC}{m}{n}{
%  <-6>    MnSymbolC5
%  <6-7>   MnSymbolC6
%  <7-8>   MnSymbolC7
%  <8-9>   MnSymbolC8
%  <9-10>  MnSymbolC9
%  <10-12> MnSymbolC10
%  <12->   MnSymbolC12%
%}{}
%\DeclareMathSymbol{\powerset}{\mathord}{MnSyC}{180}

%\usepackage[left=2cm,right=2cm,top=2cm,bottom=2cm]{geometry}
\usepackage[left=1.2in, right=1.2in, top=1in, bottom=1in]{geometry}

\newtheorem*{proposition}{Proposition}
\newtheorem{lemma}{Lemma}
\newtheorem*{claim}{Claim}

\newcommand{\lemmaautorefname}{Lemma}

\title{Extra Exercises from book}

\begin{document}

\maketitle

Below are some exercises from our book that were suggested to be done in addition to the weekly homework. The book is Introduction to Real Analysis 4\textsuperscript{th} edition by Bartle and Shebert. The section and exercise numbers refer to the book. I have worked almost all of these problems and am working to get them typed up.

\section*{Section 11.1}

\begin{enumerate}[label=\arabic*., itemsep=1.5cm]

\setcounter{enumi}{2}

\item Write out the Induction argument in the proof of part (b) of the Open Set Properties 11.1.4.

\vspace{0.5cm}

\begin{claim}
The intersection of any finite collection of open sets in $\mathbb{R}$ is open.
\end{claim}

\begin{proof}
Let $G_n := G_1 \cap G_2 \cap \ldots \cap G_n$ where $n \in \mathbb{N}$ and $G_n$ is an open set for all $n \in \mathbb{N}$. To show that $G$ is open we will use an induction argument.

\underline{Base Case:} Letting $n = 1$ we have that $G = G_1$ and $G_1$ is an open set by assumption so that $G$ is open in this case.

\underline{Induction Step:} Suppose $n \geq 1$ and $\displaystyle \bigcap_{i = 1}^{n} (G_i)$ is an open set. Then $\displaystyle \bigcap_{i = 1}^{n + 1} (G_i) = (G_{n + 1}) \cap \bigg( \bigcap_{i = 1}^{n} G_i \bigg)$. Both $G_{n + 1}$ and $\displaystyle \bigcap_{i = 1}^{n} (G_i)$ are open by assumption. In the book it is proven that the intersection of two open sets are open and so we can conclude that $\displaystyle \bigcap_{i = 1}^{n + 1} (G_i)$ is open.
\end{proof}

\setcounter{enumi}{4}

\item Show that the set $\mathbb{N}$ of natural numbers is a closed set in $\mathbb{R}$.

\vspace{0.5cm}

\begin{claim}
The set $\mathbb{N}$ of natural numbers is a closed set in $\mathbb{R}$.
\end{claim}

\begin{proof}
To show that $\mathbb{N}$ is a closed set in $\mathbb{R}$ we will show that $\mathbb{R} \setminus \mathbb{N}$ is open. Let $G_n := (n, n + 1)$ for all $n \in \mathbb{N}$. Since $G_n$ is an open interval for all $n \in \mathbb{N}$, then $G_n$ is an open set for all $n \in \mathbb{N}$. Now if we show that $\displaystyle \mathbb{R} \setminus \mathbb{N} = \bigcup_{n \in \mathbb{N}} G_n$, then we can conclude the complement of $\mathbb{N}$ in $\mathbb{R}$ is open, which means $\mathbb{N}$ is closed. Now

\begin{equation*}
\begin{split}
x \in \mathbb{R} \setminus \mathbb{N} &\iff \exists k \in \{1, 2, 3, \ldots\} \text{ such that } x \in (k, k + 1) \\
&\iff x \in G_k \text{ for some } k \in \mathbb{N} \\
&\iff x \in \bigcup_{k \in \mathbb{N}} G_k.
\end{split}
\end{equation*}
\end{proof}

Therefore $\displaystyle \mathbb{R} \setminus \mathbb{N} = \bigcup_{n \in \mathbb{N}} G_n$. Since the union of an arbitrary collection of open sets is open then $\displaystyle \bigcup_{n \in \mathbb{N}} G_n$ is open and therefore $\mathbb{R} \setminus \mathbb{N}$ is open. Thus the complement of $\mathbb{N}$ in $\mathbb{R}$ is open, so $\mathbb{N}$ is closed.
\end{enumerate}



\end{document}